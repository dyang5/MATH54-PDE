\section{Introduction}

In this paper, we discuss a partial differential equation that models the evolution over time of an age-structured population. Our equation is useful for age-structured models, demographic models where the population at time $t$ has an age distribution superimposed on it. In other words, at a given time $t$, the ages of the individuals in a population are also considered. Age-structured models are related to general physiologically-structured models, where any other variable (such as size or weight) can replace the age variable. For example, De Roos and Persson use a physiologically structured population model to model a ``size-structured consumer population feeding on a non-structured prey population," studying the qualitative and quantitative population dynamics of a planktivorous fish population \cite{deRoos_Persson}. Consequently, the ideas we discuss in this paper, though focused on age, can also be extended to other physiological structures. \\


Anderson Gray McKendrick (A.G. McKendrick) was a Scottish physician and epidemiologist. In 1926, McKendrick published a paper titled \textit{Applications of mathematics to medical problems}. He aimed to study the transfer of disease caused by interactions between people and apply mathematical modeling to epidemiology; his paper introduced a version of the McKendrick-von Foerster equation \cite{mck1925}. McKendrick's paper went relatively unseen; independently, in 1959, biophysicist Heinz von Foerster discovered the same equation when studying cell divisions \cite{keyfitz_keyfitz}. The equation itself was later named in recognition of their independent work, as what we now know as the \textbf{McKendrick–von Foerster equation}, a form of the advection equation that incorporates the mortality rate to study the population density at a given age and time. \\

In our project, we will derive the general age-structured model arising from the McKendrick-von Foerster equation \cite{logan}. We will then discuss the stable age structure to see what happens after a long period of time. Finally, after detailing various extensions of the McKendrick-von Foerster equation, we will close by highlighting its importance in helping scientists understand population models. 
