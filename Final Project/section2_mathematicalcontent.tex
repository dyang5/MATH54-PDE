\section{The McKendrick-von Foerster Equation}\setcounter{secnumdepth}{2}

\subsection{Context}


Consider a population of female organisms with age structure at time $t=0$ given by $f(a)$. Equivalently, $f(a)da$ is the number of females between age $a$ and age $a+da$ in the population. \\

Though age should technically be finite, let the domain for $a$ be $[0, \infty).$ The goal of the McKendrick-von Foerster equation is to model the age structure $u=u(a, t)$ for the population for any time $t > 0$. By definition, $u(a, t) da$ represents the number of females at time $t$ between ages $a$ and $a+da$. The total female population at time $t$ can consequently be represented as
\[ N(t) = \int_0^{\infty} u(a, t) da.\]

Note that the quantity $u(0, t)$, the number of newborns at time $t$, is not known; this quantity depends on the reproduction rate of females and the mortality rate. We define $m(a)$ to be the \textit{per capita mortality rate}, which will be given to us in any initial statement of the problem, and $b(a, t)$ to be the \textit{fecundity rate} (also known as the \textit{maternity function}), the average number of offspring per female at time $t$. 
\subsection{Formula}



The general form of the \textbf{McKendrick–von Foerster equation}, which models the population dynamics described above, is
    \[ u_t = -u_a - m(a)u. \]
    
This equation is a form of the advection equation where $m(a)$ is the per capita mortality rate defined above. Similarly, $u = u(a, t)$ represents the density of a population of age $a$ and time $t$, for nonnegative $a$ and positive $t$ (where $a$ and $t$ are measured in the same units).  More precisely, the McKendrick–von Foerster equation is the advection equation with speed one and sink term given by the mortality rate; notice that the flux is $\phi = u$, or the density of the population at that age. This means we can use a certain age, \textit{a}, and a time, \textit{t}, and then compute the density of the population of that particular age at that time. \\


We can now derive the initial conditions and boundary values for the population dynamics of the given equation. \\

Note that we have an initial condition 
\[u(a, 0)=f(a), \, \, \, a \geq 0.\]

where, as stated before, $f(a)$ represents the initial population of female organisms of age $a$ at time $t=0.$ \\

Recall that $u(a, t) $ represents the number of females at time $t$ at age $a$, and $b(a, t)$ is the average reproduction rate of females of age $a$ at time $t.$ Summing over all possible ages, from $0$ to infinity, we derive the boundary condition $u(0,t) = B(t)$ where $B(t)$ represents the total amount of offspring produced by females of all ages at time $t$. Our work tells us that

\[ u(0,t) = \int_0^{\infty}b(a,t)u(a, t) \, da, \, \, \, t>0.\]

Here, we see that $u$ is part of our boundary condition. This is known as a \emph{nonlocal boundary condition}, where the unknown solution is a part of the condition. \\

We have now derived the general age-structured model
\[ u_t = -u_a-m(a)u, \, \, \, a>0, \, t>0\]
\[ u(0,t) = \int_0^{\infty}b(a,t)u(a, t) \, da, \, \, \, t>0\]
\[ u(a, 0)=f(a), \, \, \, a \geq 0\]

extending from the McKendrick-von Foerster Equation. 

