\section{Discussion and Conclusions}

In this paper, we introduce the McKendrick-von Foerster equation, which is used in age-structured models to study the population density at a given age and time. We derive the general age-structured model from the equation and then derive the stable age structure/long-term solution to the general age-structured model. \\

Further studies of specific cases of the McKendrick-von Foerster equation and the general age-structured model lead to different behaviors in the age structure for the population. For example, when the maternity rate $m$ is constant, one can study what is known as a renewal equation (see \cite{logan} for more details). \\

While the context of our paper is for age-structured models, for which the McKendrick-von Foerster equation can be used to study areas in epistemology \cite{keyfitz_keyfitz} and biology \cite{vF1959}, the McKendrick-von Foerster equation can also be used more broadly to study other general physiologically-structured models, such as size-structured models as in \cite{deRoos_Persson}. \\

The McKendrick-von Foerster equation allows researchers to work with the complexities of population growth and structure. The equation accounts for the fact that mortality rates and fertility rates usually vary with age, and it allows for the study of populations with different age structures. However, the equation comes with a few limitations. The equation assumes homogeneous populations with no migration and that all individuals have the same mortality and fertility rates. This may not be a reasonable assumption for some populations. \\

In conclusion, the McKendrick-von Foerster equation and the systems it models make it a useful tool to study population dynamics and general demography, and we hope this paper gave a glimpse into its power and flexibility.
