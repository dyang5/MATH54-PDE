\documentclass[11pt]{article}
\usepackage{graphics}
\usepackage{amsthm}
\usepackage{amsmath}
\usepackage{amssymb}
\usepackage{listings}
\usepackage{mathtools}
\usepackage[margin=1in]{geometry}
\usepackage[shortlabels]{enumitem}


\newenvironment{solution}
  {\renewcommand\qedsymbol{$\blacksquare$}\begin{proof}[Solution]}
  {\end{proof}}
  
\newcommand{\N}{\mathbb{N}}
\setlength\parindent{0pt}
\begin{document}

	\hrule
	\begin{center}
		{\Large Homework 11} \\ % Replace with the homework number
		\vspace{0.2cm}
		Partial Differential Equations, Spring 2023 \hfill David Yang % Replace with your name(s)
	\end{center}

\hrule

\vspace{1em}


\underline{Chapter 4.7, Example 4.28} \\

\textbf{Consider the problem}

\[ u_t - 3u_{xx} = 0, \, \, \, 0 < x < 1, \, t > 0,\]
\[ u(0, t) = 2e^{-t}, \, \, \, u(1, t) = 1\]
\[ u(x, 0) = x^2, \, \, \, 0 < x < 1.\]
\textbf{Complete the calculation (Solve the $w$ PDE using the eigenfunction method).}

\begin{solution}
Following Example 4.28, we homogenize the boundary condition by defining \[ w(x, t) = u(x, t) - \left( 2e^{-t} + \left(1-2e^{-t}\right)x\right).\]

Then $w$ solves the problem 

\[ w_t - 3u_{xx} = 2e^{-t}(1-x), \, \, \, 0 < x < 1, \, t > 0,\]
\[ w(0, t) = w(1, t), \, \, \, t > 0\]
\[ w(x, 0) = x^2+x, \, \, \, 0 < x < 1.\]
\end{solution}

\underline{Chapter 4.7, Exercise 7} \\

\textbf{Solve twice and check your answers match:}

\begin{enumerate}[(a)]
    \item \textbf{Method 1: Apply the eigenfunction method directly to the non-homogeneous PDE for $u$.}
    \item \textbf{Method 2: Observe that the source term is time-independent. Convert the PDE to a homogeneous PDE for $w = u - u_{ss}$ where $u_{ss}$ is the steady state solution to the PDE.
    (see Remark 4.29 on page 212). Solve the homogeneous PDE for $w$ and recover $u$ as $u = w + u_{ss}$.}
\end{enumerate}

\newpage

\underline{Chapter 4.2, Exercise 5} \\

\textbf{For the SLP (Sturm-Liouville Problem)} \[-y'' = \lambda y, \, \, \, 0 < x < l; \, \, \, y(0) - ay'(0) = 0, \, \, \, y(l) + by'(l) = 0,\]
\textbf{show that $\lambda = 0$ is an eigenvalue if and only if $a+b = -l.$} \\

\begin{solution}
If $\lambda = 0$ is an eigenvalue, we have that $-y''(x) = 0$, so $y''(x) = 0$ and \[y(x) = C_1x + C_2.\]

Furthermore, plugging in $x = 0$, we find that $y(0) = C_2$ and $y'(0) = C_1$. Plugging these into the first boundary condition, we get that
\[ C_2 - aC_1 = 0.\]

Similarly, plugging in $x = l$, we find that $y(l) = C_1 l + C_2$ and $y'(l) = C_1$. Plugging these into the second boundary condition, we get that 
\[ C_1l + C_2 + bC_1 = (b+l) C_1 + C_2 = 0.\]

We are left with the system of equations
\[ \begin{cases}
    -aC_1 + C_2 = 0 \\
    (b+l)C_1 + C_2 = 0
\end{cases}. \]

Solving for $C_1$ by subtracting the two equations, we find that
\[ C_1(-a-b-l) = 0.\]

Since we must have that $C_1$ and $C_2$ are not both $0$, we know that the SLP has eigenvalue $0$ if and only if $-a-b-l = 0$, or when $a+b = -l$, as desired.
\end{solution}

\newpage

\underline{Chapter 4.2, Exercise 9} \\

\textbf{Find the eigenvalues and eigenfunctions for the following problem with \textit{periodic} boundary conditions:}
\[ -y''(x) = \lambda y(x), \, \, \, 0 < x < l, \]
\[ y(0) = y(l), y'(0) = y'(l).\]

\begin{solution}
We will split our work into three cases: when $\lambda = 0, \lambda < 0$, and $\lambda > 0$. \\

First, if $\lambda = 0$, then we find that $y''(x) = 0$, so $y = ax + b$ for constants $a, b$. However, if $y(0) = y(l)$, then we must have that $a = 0$.
There are no further restrictions on the constant $b$, so our boundary conditions tell us that the eigenvalue $\boxed{\lambda = 0 \text{ corresponds to a constant eigenfunction}}.$ \\

Next, if $\lambda < 0$, then $y''(x) + \lambda y(x) = 0$ has solution \[ y(x) = ae^{-\sqrt{\lambda} x} + be^{\sqrt{\lambda x}}. \]
As we've shown before, exponential solutions cannot satisfy periodic boundary conditions, and so we have a trivial solution in this case. \\

Finally, we consider the case when $\lambda > 0$. The ODE $y''(x) + \lambda y(x) = 0$ will then have solution
\[ y(x) = a\cos(\sqrt{\lambda}x) + b\sin(\sqrt{\lambda}x).\]

The boundary condition $y(0) = y(l)$ tells us that \[ b = a\sin(\sqrt{\lambda} l) + b\cos(\sqrt{\lambda} l)\]

and the boundary condition $y'(0) = y'(l)$ tells us that \[ a\sqrt{\lambda} = a\sqrt{\lambda}\cos(\sqrt{\lambda} l) - b\sqrt{\lambda}\sin(\sqrt{\lambda} l)\]

Thus, after simplification, our boundary conditions give us the following sytem of equations:
\[ \begin{cases}
a\sin(\sqrt{\lambda} l) + b(\cos(\sqrt{\lambda} l) - 1) = 0 \\
a\cos(\sqrt{\lambda} l - 1) - b(\sin(\sqrt{\lambda} l)) = 0
\end{cases}.\]

Rewriting this system as a matrix expression, we have that
\[ 
\begin{bmatrix}
\sin(\sqrt{\lambda} l) & \cos(\sqrt{\lambda} l) - 1 \\
\cos(\sqrt{\lambda} l) - 1 & -\sin(\sqrt{\lambda} l) \\
\end{bmatrix}
\begin{bmatrix}
a \\ 
b
\end{bmatrix}
=
\begin{bmatrix}
0 \\ 0
\end{bmatrix}.\]

This system only has a nontrivial eigenfunction if $a$ and $b$ are not both $0$. Equivalently, we must have that
\[ \mathrm{det} \left( \begin{bmatrix}
    \sin(\sqrt{\lambda} l) & \cos(\sqrt{\lambda} l) - 1 \\
    \cos(\sqrt{\lambda} l) - 1 & -\sin(\sqrt{\lambda} l) \\
    \end{bmatrix} \right) = 2\cos(\sqrt{\lambda} l) = 0. \]

Since $\cos(\lambda l) = 0$, we must have that $\lambda l = 2\pi n$ for integer $n$, and so we have that the eigenvalues 
\[\boxed{\lambda = \left( \frac{2\pi n}{l}\right)^2}\] 

correspond to eigenfunctions \[ \boxed{y(x) = a_n \sin \left( \frac{2\pi n}{l} x\right) + b_n \cos \left( \frac{2\pi n}{l} x\right)}.\]
in the given problem. \end{solution}

\newpage

\underline{Chapter 4.4, Exercise 1} \\
\end{document}