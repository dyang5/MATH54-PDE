\documentclass[11pt]{article}
\usepackage{graphics}
\usepackage{amsthm}
\usepackage{amsmath}
\usepackage{amssymb}
\usepackage{listings}
\usepackage{mathtools}
\usepackage[margin=1in]{geometry}
\usepackage{enumerate}

\newenvironment{solution}
  {\renewcommand\qedsymbol{$\blacksquare$}\begin{proof}[Solution]}
  {\end{proof}}
  
\newcommand{\N}{\mathbb{N}}
\setlength\parindent{0pt}
\begin{document}

	\hrule
	\begin{center}
		{\Large Homework 5} \\ % Replace with the homework number
		\vspace{0.2cm}
		Partial Differential Equations, Spring 2023 \hfill David Yang % Replace with your name(s)
	\end{center}

\hrule

\vspace{1em}

\underline{HW 5 Problem} \\

\textbf{Consider the PDE \[u_{xx} - 3u_{xt} - 4u_{tt} = 0 \, \, \, \, \, \text{ for } x \in \mathbb{R} \text { and } t > 0\] with initial conditions 
\[u(x, 0) = x^2 \, \, \, \, \, \text{ and } \, \, \, \, \, u_t(x, 0) = e^x \text{ for } x \in \mathbb{R}\]}
\begin{enumerate}[a)]
\item \textbf{Calculate the discriminant and classify the PDE as hyperbolic, parabolic, or elliptic.} 
\begin{solution}
This PDE is a second-order differential equation of the form $Au_{xx} + Bu_{xt} + Cu_{tt} = 0$, with $A = 1$, $B = -3$, and $C=-4.$ \\

The discriminant of the PDE, then, is \[D = B^2 - 4AC = (-3)^2 - 4(1)(-4) = \boxed{25} > 0.\]
Since this PDE has a positive discriminant, the PDE is \boxed{\text{hyperbolic}}.
\end{solution}

\item \textbf{Solve the PDE.}
\begin{solution}
We will make a change of variables of the form 
\[\begin{cases}
\xi = ax + bt \\
\tau = cx+dt
\end{cases},\]

as done in Logan, page 75. Since the PDE is hyperbolic, we can follow the procedure on page 75 once again. We choose $ a = c = 1$, and get that
\[ b = \frac{-B + \sqrt{D}}{2C} = \frac{-(-3) + \sqrt{25}}{2(-4)} = -1 \]
\[ d =  \frac{-B - \sqrt{D}}{2C} = \frac{-(-3) - \sqrt{25}}{2(-4)} = \frac{1}{4}. \]

Thus, our change of coordinates are 
\[\begin{cases}
	\xi = x - t \\
	\tau = x + \frac{1}{4}t
	\end{cases}.\]

Since our original PDE had no lower order terms, this change of coordinates transforms our PDE into the canonical form \[U_{\xi\tau} = 0.\]

We can now simply integrate twice to solve for $U$. First, integrating both sides with respect to $\tau$, we get that
\[U_{\xi} = \tilde{f}(\xi) \]

where $\tilde{f}(\xi)$ is a function of $\xi.$ Integrating again, but this time with respect to $\xi$, we get that
\[U = f(\xi) + g(\tau), \]
where $f(\xi) = \int \tilde{f} \, d\xi$ is simply another function of $\xi$. \\

Our change of coordinates was $\xi = x - t$ and $\tau = x + \frac{1}{4}t$. Reverting our solution back to $x-t$ coordinates, we have that
\[ u(x, t) = f(x-t) + g\left(x+\frac{1}{4}t\right). \]

We can now use our initial conditions to determine the functions $f$ and $g$. First, we are given that $u(x, 0) = x^2$, so we know that
\begin{align*}
	u(x, 0) = f(x-0) + g\left(x + \frac{1}{4}\cdot 0 \right) = f(x) + g(x) = x^2
\end{align*}

Similarly, taking the partial derivative of $u(x, t)$ with respect to $t$, we get that
\[ u_t(x, t) = -f'(x-t) + \frac{1}{4}g'\left(x+\frac{1}{4}t \right)\]

Thus, since $u_t(x, 0) = e^x$, we have that
\[ u_t(x, 0) = -f'(x-0) + \frac{1}{4}g'\left(x+\frac{1}{4}\cdot0 \right) = -f'(x) + \frac{1}{4}g'(x) = e^x.\]

Combining the information from the initial conditions, we have the system of equations
\[\begin{cases}
f(x) + g(x) = x^2 \\
-f'(x) + \frac{1}{4}g'(x) = e^x
\end{cases}\]

Taking the derivative of the first equation, we get the new system
\[\begin{cases}
	f'(x) + g'(x) = 2x \\
	-f'(x) + \frac{1}{4}g'(x) = e^x
\end{cases}\]

We can now solve for $f'(x)$ and $g'(x)$. To solve for $g'(x)$, we add the two equations and multiply both sides by $\frac{4}{5}$ to get 
\[g'(x) = \frac{4}{5} \left(2x+e^x\right) = \frac{8x}{5} + \frac{4}{5}e^x.\]

We can then get $f'(x)$ from the first equation. Since $f'(x) = 2x - g'(x)$, we have \[f'(x) = \frac{2x}{5} - \frac{4}{5}e^x.\]

We can now integrate these two equations to get $f(x)$ and $g(x)$. We find that
\[f(x) = \frac{x^2}{5} -\frac{4}{5}e^x \, \, \, \text { and } \, \, \, g(x) = \frac{4x^2}{5} + \frac{4}{5}e^x. \]

From our work before, we have that
\[ u(x, t) = f(x-t) + g\left( x + \frac{1}{4}t \right).\]

Substituting the equations we found for $f(x)$ and $g(x)$, we get that
\begin{align*}u(x, t) &= f(x-t) + g\left( x + \frac{1}{4}t \right) \\
	&= \left(\frac{(x-t)^2}{5} -\frac{4}{5}e^{x-t} \right) + \left(\frac{4(x+\frac{1}{4}t)^2}{5}+ \frac{4}{5}e^{x+\frac{1}{4}t}\right)
\end{align*}

and so our solution that satisfies the original PDE and its initial conditions is 
\[ \boxed{u(x, t) = \left(\frac{(x-t)^2}{5} -\frac{4}{5}e^{x-t} \right) + \left(\frac{4(x+\frac{1}{4}t)^2}{5}+ \frac{4}{5}e^{x+\frac{1}{4}t}\right) }.\]
\end{solution} 
\end{enumerate}



\end{document}