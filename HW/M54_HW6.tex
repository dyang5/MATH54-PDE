\documentclass[11pt]{article}
\usepackage{graphics}
\usepackage{amsthm}
\usepackage{amsmath}
\usepackage{amssymb}
\usepackage{listings}
\usepackage{mathtools}
\usepackage[margin=1in]{geometry}
\usepackage{enumerate}

\newenvironment{solution}
  {\renewcommand\qedsymbol{$\blacksquare$}\begin{proof}[Solution]}
  {\end{proof}}
  
\newcommand{\N}{\mathbb{N}}
\setlength\parindent{0pt}
\begin{document}

	\hrule
	\begin{center}
		{\Large Homework 6} \\ % Replace with the homework number
		\vspace{0.2cm}
		Partial Differential Equations, Spring 2023 \hfill David Yang % Replace with your name(s)
	\end{center}

\hrule

\vspace{1em}

\underline{HW 6 Problems} 
\begin{enumerate}
\item \textbf{Find a solution to the diffusion PDE} \[ u_t - u_{xx} = 0 \, \, \, \text{for } x \in \mathbb{R}, \, t > 0\] \textbf{with initial value} 
\[ u(x, 0) = e^{-x^2/4} \, \, \, \text{for } x \in \mathbb{R}.\]

\begin{solution}
We will use the solution formula for the diffusion equation on an unbounded domain. Recall that the solution to the diffusion PDE $u_t - ku_{xx} = 0$ on the unbounded domain with initial value $u(x, 0) = \phi(x)$ is
is \[ u(x, t) = \frac{1}{\sqrt{4\pi k t}} \int_{-\infty}^{\infty} \phi(y) e^{-(x-y)^2/(4kt)} \, dy.\]

Applying the solution formula to this problem (where $k=1$ and $\phi(x) = e^{-x^2/4}$), we have that the solution to the given diffusion PDE on the unbounded domain is
\[ u(x, t) = \frac{1}{\sqrt{4\pi t}} \int_{-\infty}^{\infty} e^{-y^2/4} e^{-(x-y)^2/(4t)} \, dy.\]

Simplifying the integrand by combining the exponential terms, we get that
\begin{align*}u(x, t) &= \frac{1}{\sqrt{4\pi t}} \int_{-\infty}^{\infty} e^{-ty^2/(4t)} e^{-(x-y)^2/(4t)} \, dy \\
&= \frac{1}{\sqrt{4\pi t}} \int_{-\infty}^{\infty} e^{\frac{-(x-y)^2-ty^2}{4t}}\, dy\end{align*}

Expanding the numerator and then completing the square, we get that
\begin{align*}u(x, t) &= \frac{1}{\sqrt{4\pi t}} \int_{-\infty}^{\infty} e^{\frac{-(x-y)^2-ty^2}{4t}} \, dy \\
	&= \frac{1}{\sqrt{4\pi t}} \int_{-\infty}^{\infty} e^{-\frac{(x^2-2xy+(t+1)y^2)}{4t}}\, dy \\
	&= \frac{1}{\sqrt{4\pi t}} \int_{-\infty}^{\infty} e^{-\frac{(t+1)\left(y-\frac{x}{t+1}\right)^2 + x^2 \cdot \frac{t}{t+1}}{4t}}\, dy.\end{align*}

Factoring out a $e^{-\frac{x^2 \cdot \frac{t}{t+1}}{4t}} = e^{-\frac{x^2}{4(t+1)}}$ term from the integrand, we get that
\[ u(x, t) = \frac{1}{\sqrt{4\pi t}} e^{-\frac{x^2}{4(t+1)}}\int_{-\infty}^{\infty} e^{-\frac{(t+1)\left(y-\frac{x}{t+1}\right)^2}{4t}}\, dy \]

We will now make a substitution to transform the integrand into $e^{-r^2}$: let \[r = \frac{\left(y-\frac{x}{t+1}\right)\sqrt{t+1}}{\sqrt{4t}}.\]
Then we also have that \[ dr = \frac{\sqrt{t+1}}{\sqrt{4t}} \, dy. \]

Making the substitution for $r$ in our solution $u(x, t)$, we find that 
\begin{align*} u(x, t) &= \frac{1}{\sqrt{4\pi t}} e^{-\frac{x^2}{4(t+1)}}\int_{y = -\infty}^{y = \infty} e^{-\frac{(t+1)\left(y-\frac{x}{t+1}\right)^2}{4t}}\, dy \\
&= \frac{1}{\sqrt{\pi(t+1)}} e^{-\frac{x^2}{4(t+1)}} \int_{r = -\infty}^{r = \infty} e^{-r^2} \, dr. \end{align*}

But we also know that $\int_{-\infty}^{\infty} e^{-r^2} \, dr = \sqrt{\pi}$, so plugging this back into our solution, we find that
\begin{align*} u(x, t) &= \frac{1}{\sqrt{\pi(t+1)}} e^{-\frac{x^2}{4(t+1)}} \int_{r = -\infty}^{r = \infty} e^{-r^2} \, dr \\
&= \left(\frac{1}{\sqrt{\pi(t+1)}} e^{-\frac{x^2}{4(t+1)}}\right) \cdot \sqrt{\pi} \\
&= \frac{1}{\sqrt{t+1}} e^{-\frac{x^2}{4(t+1)}}. \end{align*}

Thus, our solution to the given diffusion PDE with the given initial value is
\[ \boxed{u(x, t) = \frac{1}{\sqrt{t+1}} e^{-\frac{x^2}{4(t+1)}}.}\]
	

\end{solution}

\newpage

\item \textbf{Show that your solution to \# 1 satisfies the property that, for all $t > 0$,}
\[ \int_{-\infty}^{\infty} u(x, t) \, dx = \int_{-\infty}^{\infty} u(x, 0) \, dx.  \]
\textbf{In other words, $\int_{-\infty}^{\infty} u(x, t) \, dx$ is a \textit{conserved quantity} (constant with respect to $t$). }

\begin{solution}To show that this property holds, we will show that the left and right hand sides simplify to the same expression. We will begin by working with the left-hand side. \\

Plugging in our solution for $u(x, t)$ from the first problem into the left-hand side of the equation, we have that
\[ \int_{-\infty}^{\infty} u(x, t) \, dx = \int_{-\infty}^{\infty}  \frac{1}{\sqrt{t+1}} e^{-\frac{x^2}{4(t+1)}} \, dx.\]

Let us make the substitution $y = \frac{x}{\sqrt{4(t+1)}}.$ We also have that $dy = \frac{1}{\sqrt{4(t+1)}} \, dx.$ Substituting this into our integral, we get that
\begin{align*} \int_{-\infty}^{\infty} u(x, t) \, dx &= \int_{-\infty}^{\infty}  \frac{1}{\sqrt{t+1}} e^{-\frac{x^2}{4(t+1)}} \, dx \\
&= 2 \int_{-\infty}^{\infty}  \frac{1}{\sqrt{4(t+1)}} e^{-\frac{x^2}{4(t+1)}} \, dx \\
&= 2 \int_{y= -\infty}^{y=\infty} e^{-y^2} \, dy.\end{align*}

We will now show that the right-hand side $\int_{-\infty}^{\infty} u(x, 0) \, dx$ can be rewritten in the same form. Plugging in the initial condition $u(x,0) = e^{-x^2/4}$ into this integral, we get
\[ \int_{-\infty}^{\infty} u(x, 0) \, dx = \int_{-\infty}^{\infty} e^{-\frac{x^2}{4}} \, dx.\]

Let us now make the substitution $y = \frac{x}{2}$. This also gives us $dy = \frac{1}{2} \, dx$. Substituting this back into our integral, we get
\begin{align*}  \int_{-\infty}^{\infty} u(x, 0) \, dx &= \int_{-\infty}^{\infty}  e^{-\frac{x^2}{4}} \, dx \\
&= 2 \int_{-\infty}^{\infty}  \frac{1}{2}e^{-\frac{x^2}{4}} \, dx \\
&= 2 \int_{y=-\infty}^{y=\infty}  e^{-y^2} \, dy.
\end{align*}

Thus, we see that \[ \int_{-\infty}^{\infty} u(x, t) \, dx = \int_{-\infty}^{\infty} u(x, 0) \, dx =  2 \int_{-\infty}^{\infty}  e^{-y^2} \, dy = 2\sqrt{\pi}. \]

and so we find that $ \int_{-\infty}^{\infty} u(x, t) \, dx$ is a \textit{conserved quantity} (constant with respect to $t$), with $\int_{-\infty}^{\infty} u(x, t) \, dx = 2\sqrt{\pi}$ for all $t \geq 0$, as desired. \end{solution}


\newpage

\item \begin{enumerate}
	
\item \textbf{If $u$ solves the diffusion equation on the infinite domain $(x \in \mathbb{R})$, with bounded initial value $u(x, 0) = \phi(x)$ that has the property that}
\[ \lim_{x \rightarrow -\infty} \phi(x) = a \, \, \, \textbf{and} \, \, \, \lim_{x \rightarrow \infty} \phi(x) = b \, \, \, \, \, \,\textbf{(a, b constants)}.\]

\textbf{What is the value of $\lim_{t\rightarrow \infty} u(x, t)$?}

\begin{solution}
$u$ is a solution to the diffusion equation on the infinite domain with bounded initial value $u(x, 0) = \phi(x)$. Consequently, the solution (in the Poisson Integral representation form) is 
\[ u(x, t) = \frac{1}{\sqrt{\pi}} \int_{-\infty}^{\infty} e^{-r^2} \phi(x-r\sqrt{4kt}) \, dr.\]

Splitting the integral up into two separate integrals, we get that
\[ u(x, t) = \frac{1}{\sqrt{\pi}} \left[\int_{-\infty}^{0} e^{-r^2} \phi(x-r\sqrt{4kt}) \, dr + \int_{0}^{\infty} e^{-r^2} \phi(x-r\sqrt{4kt}) \, dr \right] \]

Thus,\[ \lim_{t \rightarrow \infty} u(x, t) = \lim_{t \rightarrow \infty} \frac{1}{\sqrt{\pi}} \left[\int_{-\infty}^{0} e^{-r^2} \phi(x-r\sqrt{4kt}) \, dr + \int_{0}^{\infty} e^{-r^2} \phi(x-r\sqrt{4kt}) \, dr \right]. \]

Per the given hint, we can pass the limit with respect to $t$ inside the integral, so we get that
\[ \lim_{t \rightarrow \infty} u(x, t) = \frac{1}{\sqrt{\pi}} \left[\int_{-\infty}^{0} \lim_{t \rightarrow \infty} e^{-r^2} \phi(x-r\sqrt{4kt}) \, dr + \int_{0}^{\infty} \lim_{t \rightarrow \infty} e^{-r^2} \phi(x-r\sqrt{4kt}) \, dr \right]. \]

We can evaluate each of the limits separately: note that when $r < 0$ (corresponding to the first integral), we have that
\begin{align*}\lim_{t \rightarrow \infty} e^{-r^2} \phi(x-r\sqrt{4kt}) \, dr &= e^{-r^2} \lim_{t \rightarrow \infty} \phi(x-r\sqrt{4kt}) \\
&= e^{-r^2} \lim_{y \rightarrow \infty} \phi(y) \\
&= e^{-r^2} b.\end{align*}

Similarly, we have that when $r > 0$ (corresponding to the second integral),
\begin{align*}\lim_{t \rightarrow \infty} e^{-r^2} \phi(x-r\sqrt{4kt}) \, dr &= e^{-r^2} \lim_{t \rightarrow \infty} \phi(x-r\sqrt{4kt}) \\
	&= e^{-r^2} \lim_{y \rightarrow -\infty} \phi(y) \\
	&= e^{-r^2} a.\end{align*}

Plugging these results back into our equation, we find that
\begin{align*} \lim_{t \rightarrow \infty} u(x, t) &= \frac{1}{\sqrt{\pi}} \left[\int_{-\infty}^{0} \lim_{t \rightarrow \infty} e^{-r^2} \phi(x-r\sqrt{4kt}) \, dr + \int_{0}^{\infty} \lim_{t \rightarrow \infty} e^{-r^2} \phi(x-r\sqrt{4kt}) \, dr \right] \\
&= 	\frac{1}{\sqrt{\pi}} \left[\int_{-\infty}^{0} e^{-r^2} b \, dr + \int_{0}^{\infty} e^{-r^2} a  \, dr  \right]
\end{align*}

Factoring out the constants $a$ and $b$ and using the fact that \[\int_{-\infty}^{0} e^{-r^2} \, dr = \int_{0}^{\infty} e^{-r^2} \, dr = \frac{\sqrt{\pi}}{2},\] we have that
\begin{align*} \lim_{t \rightarrow \infty} u(x, t) &= \frac{1}{\sqrt{\pi}} \left[\int_{-\infty}^{0} e^{-r^2} b \, dr + \int_{0}^{\infty} e^{-r^2} a  \, dr  \right] \\
&= \frac{1}{\sqrt{\pi}} \left[ b \int_{-\infty}^{0} e^{-r^2} \, dr + a \int_{0}^{\infty} e^{-r^2} \, dr \right] \\
&= \frac{1}{\sqrt{\pi}} \left[b \frac{\sqrt{\pi}}{2} + a \frac{\sqrt{\pi}}{2} \right] \\
&= \frac{1}{2} (a+b).
\end{align*}

Thus, we have that \[\boxed{\lim_{t\rightarrow \infty} u (x, t) = \frac{1}{2} (a+b)}.\]
\end{solution}




\item \textbf{Review Eq 2.5 on page 82 of Logan, which is a solution for the PDE}
\[ w_t = kw_{xx} \, \, \, \text{for } x \in \mathbb{R}, \, t > 0\]
\[ w(x, 0) = 0 \, \, \,\text{for } x < 0 \, ; \, w(x, 0) = 1 \, \, \, \text{for } x > 0.\]

\textbf{What is $\lim_{t \rightarrow \infty} w(x, t)$ and does this agree with your result in 3(a)?}

\begin{solution}
Eq 2.5 on page 82 of Logan gives us a solution to the given PDE:
\[ w(x, t) = \frac{1}{2} + \frac{1}{\sqrt{\pi}} \int_0^{x/(\sqrt{4kt})} e^{-r^2} \, dr.\]

Taking the limit of $w(x, t)$ as $t$ approaches $\infty$, we find that 
\[\lim_{t \rightarrow \infty} w(x, t) = \frac{1}{2} + \frac{1}{\sqrt{\pi}} \lim_{t \rightarrow \infty} \int_0^{x/(\sqrt{4kt})} e^{-r^2} \, dr.\]

Since $\lim_{t \rightarrow \infty} \frac{x}{\sqrt{4kt}} = 0$, we have that 
\begin{align*} \lim_{t \rightarrow \infty} w(x, t) &= \frac{1}{2} + \frac{1}{\sqrt{\pi}} \lim_{t \rightarrow \infty} \int_0^{x/(\sqrt{4kt})} e^{-r^2} \, dr \\
&= \frac{1}{2} + \frac{1}{\sqrt{\pi}} \int_0^0 e^{-r^2} \, dr \\
&= \frac{1}{2} + \frac{1}{\sqrt{\pi}} \cdot 0 \\ &= \frac{1}{2}.\end{align*}

Since $\lim_{x \rightarrow -\infty} w(x, 0) = 0$ and $\lim_{x \rightarrow \infty} w(x, 0) = 1$, and 
\[\lim_{t \rightarrow \infty} w(x, t)= \frac{1}{2} = \frac{1}{2} (0 + 1),\]

we see that our answer does match up with our result from 3(a), as desired.
\end{solution}
\end{enumerate}
\end{enumerate}




\end{document}