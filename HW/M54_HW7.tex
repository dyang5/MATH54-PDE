\documentclass[11pt]{article}
\usepackage{graphics}
\usepackage{amsthm}
\usepackage{amsmath}
\usepackage{amssymb}
\usepackage{listings}
\usepackage{mathtools}
\usepackage[margin=1in]{geometry}

\newenvironment{solution}
  {\renewcommand\qedsymbol{$\blacksquare$}\begin{proof}[Solution]}
  {\end{proof}}
  
\newcommand{\N}{\mathbb{N}}
\setlength\parindent{0pt}
\begin{document}

	\hrule
	\begin{center}
		{\Large Homework 7} \\ % Replace with the homework number
		\vspace{0.2cm}
		Partial Differential Equations, Spring 2023 \hfill David Yang % Replace with your name(s)
	\end{center}

\hrule

\vspace{1em}


\underline{Chapter 2.6, Problem 7} \\

\textbf{Solve $u_t =u_{xx}$ on $x,t>0$ with $u(0,t)=a, \, t>0$ and $u(x,0)=b$, where $a$ and $b$ are constants.} \\

\begin{solution}
We begin by taking the Laplace transform of the PDE $u_t = u_xx.$ This gives us $ \mathcal{L}(u_t) = \mathcal{L}(u_{xx}).$ Equivalently, we have that
\[ s\mathcal{L}[u(x, s)] - u(x, 0) = \mathcal{L}[u_{xx}],\]

or \[sU(x, s) - b = U_{xx}(x, s).\]

We now have a simple second-order linear ODE that we can solve: $U_{xx} - sU = -b.$ To find the general solution for this ODE, we will add the solution to the homogeneous ODE with a particular solution. \\

We can first solve the homogeneous ODE. Note that the characteristic polynomial for the homogeneous ODE  $U_{xx} - sU = 0$ is $r^2 - s =0$, with solutions $r = \pm \sqrt{s}.$ This tells us that the homogeneous solution is \[U(x, s) = a(s)e^{-x\sqrt{s}} + b(s)e^{-x\sqrt{s}}.\]
However, since we want $U$ to be bounded, we also know that $b(s) = 0$. Thus, \[U(x, s) = a(s)e^{-x\sqrt{s}}\] is our bounded solution for the homogeneous ODE. \\

We can now notice that $U(x, s) = \frac{b}{s}$ is a particular solution to the ODE, as $U_{xx} = 0$ and $-sU = -b$, so $U_{xx} - sU = -b.$ \\

Thus, combining our homogeneous and particular solutions gives us our general solution:
\[ U(x, s) = a(s)e^{-x\sqrt{s}} + \frac{b}{s}. \]

We can now apply our boundary condition $u(0, t) = a.$ This condition tells us that \[U(0, s) = \mathcal{L}(a) = \frac{a}{s}.\]
Plugging in $x=0$ into the general solution we found earlier, we have that $U(0, s) = a(s) + \frac{b}{s}$. Thus, we must have that
\[a(s) + \frac{b}{s} = \frac{a}{s} \]
and so we know that \[a(s) = \frac{a-b}{s}.\]

Plugging this back into our general equation gives us the solution in the transform domain:
\begin{align*} U(x, s) &= a(s)e^{-x\sqrt{s}} + \frac{b}{s} \\ &= \frac{a-b}{s}e^{-x\sqrt{s}} + \frac{b}{s}.\end{align*}

Applying the inverse Laplace transform to this solution gives us the solution to our original problem:
\[\boxed{u(x, t) = (a-b)\left( 1 - \mathrm{erf}\left( \frac{x}{2\sqrt{t}}\right)\right) + b}.\]
\end{solution}


\newpage


\underline{Chapter 2.7, Problem 15} \\

\textbf{Solve the Cauchy problem for the advection-diffusion equation using Fourier transforms:}
\[u_t = Du_{xx} - cu_x, \, \, \, x \in \mathbb{R}, \, t>0; \, \, \, u(x,0)=\phi(x), x\in\mathbb{R}.\]


\end{document}