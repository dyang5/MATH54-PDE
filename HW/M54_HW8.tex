\documentclass[11pt]{article}
\usepackage{graphics}
\usepackage{amsthm}
\usepackage{amsmath}
\usepackage{amssymb}
\usepackage{listings}
\usepackage{mathtools}
\usepackage[margin=1in]{geometry}
\usepackage[shortlabels]{enumitem}


\newenvironment{solution}
  {\renewcommand\qedsymbol{$\blacksquare$}\begin{proof}[Solution]}
  {\end{proof}}
  
\newcommand{\N}{\mathbb{N}}
\setlength\parindent{0pt}
\begin{document}

	\hrule
	\begin{center}
		{\Large Homework 8} \\ % Replace with the homework number
		\vspace{0.2cm}
		Partial Differential Equations, Spring 2023 \hfill David Yang % Replace with your name(s)
	\end{center}

\hrule

\vspace{1em}


\underline{HW 8 Problem} \\

\textbf{Consider the 1st order linear initial value PDE problem:
\[tu_t + u_x = 0 \, \, \, \text{for } t > 0, x \in \mathbb{R}\]
\[ u(x, 0) = f(x) \, \, \, \text{for } x \in \mathbb{R}.\]}

\begin{enumerate}[(a)]
    \item \textbf{Apply the Method of Characteristics. Your goal is to find a characteristic value $\xi(x, t)$ so that any function of the form $u(x, t) = f(\xi)$ satisfies the PDE.} \\
     
    \textbf{Tip: In most problems of this type that you have worked on in before, you have set $t = \tau$
    and used the $\xi$ variable to parameterize the values of $x$ along the $x$ axis. For this problem,
    set $x=\tau$ and use the $\xi$ variable to parameterize the values of $t$ along the $t$-axis.}
    
    \begin{solution}
    We begin by applying the Method of Characteristics. We get that 
    \[ t_{\tau} = t, \, \, \, x_{\tau} = 1, \text{ and } U_{\tau} = 0.\]

    Since, as the tip says, we set $x = \tau$ and use the $\xi$ variable to parameterize the values of $t$ along the $t$-axis, we know 
    \[ t(\xi = 0) = 0 \, \, \, \text{and} \, \, \, x(\xi = 0) = \tau.\]

    Solving for $t$ and $x$ by integrating with respect to $\tau$ and using these initial values, we find that 
    \[ t = \xi e^{\tau} \, \, \, \text{ and } \, \, \, x = \tau.\]

    Solving for $U$, we get that \[ U = f(\xi)\] which also matches the given initial condition $U(\xi, 0) = f(\xi).$ \\

    Now, solving for $\xi$ and $\tau$ in terms of $x$ and $t$, we get that \[ \tau = x \, \, \, \text{ and } \, \, \, \xi = te^{-x}.\]

    Plugging this change of variables back into our solution, we get that
    \[u(x(\xi, \tau), t(\xi, \tau)) = f(\xi(x, t)) = f(te^{-x}).\]

    Thus, our solution is \[\boxed{u(x, t) = f(te^{-x})}.\]
    
    \end{solution}
    \item \textbf{Set $f(x) = x$ as the initial value for the PDE given above. Show that the form of the solution you found in (a) does not satisfy this initial value. } \\
     
    \textbf{\textit{Remark: In fact, no solutions exist for this PDE that solve the initial value $u(x, 0) = x$. This PDE is ill-posed for $u(x, 0) = x$.}}

    \begin{solution}
      We set $f(x) = x$ as per the instructions. We will now check whether the solution $u(x, t) = f(te^{-x})$ satisfies the initial value
      \[ u(x, 0) = x \, \, \, \text{for } x \in \mathbb{R}.\]
  
      Plugging in $t=0$ to our solution, we get that \[ u(x, 0) = 0 \text{ for } x \in \mathbb{R} \]
  
      However, notice that to satisfy the initial condition, we must have that $u(x, 0) = f(x) = x$ for $x\in \mathbb{R}$. \\
  
      Clearly, \[ u(x, 0) = 0 \neq x \text{ for } x \in \mathbb{R}, \] and so the solution we found in (a) does not satisfy the initial value for the PDE.
      \end{solution}
    
    \item \textbf{Set $f(x) = 1$ as the initial value for the PDE given above. Now let $\xi$ be the characteristic variable you found in (a). 
    For what values of the constants $a$ and $b$ does the function $u(x, t) = a + b\xi$ also solve the PDE and satisfy the initial value $u(x,0) = 1$?}
    
    \begin{solution}
    Let $u(x, t) = a + b\xi$, where $\xi = te^{-x}$ as we found in part (a). Equivalently, we have that
    \[ u(x, t) = a + bte^{-x}.\]

    Calculating the partial derivatives, we get that \[ u_t = a + be^{-x} \,\,\, \text { and } \,\,\, u_x = a-bte^{-x}.\]

    Plugging these partials back into our PDE, we get
    \begin{align*} tu_t + u_x &= t\left( a+be^{-x}\right) + \left(a-bte^{-x}\right) \\
      &= \left(at + bte^{-x}\right) + \left(a - bte^{-x} \right) \\
      &= at + a.
    \end{align*}

    Thus, to satisfy the initial PDE problem $tu_t + u_x = 0$ for all $t  > 0, x \in \mathbb{R}$, we must have that \[at + a = 0\] for all $t > 0$, meaning that $a$ must be $0$. \\

    On the other hand, to satisfy the initial value $u(x, 0) = 1$, we must have
    \begin{align*} u(x, 0) &= a + b(0)e^{-x} \\
      &= a \\
      &= 1.\end{align*}
    \end{solution}
    \item \textbf{Is the PDE with the initial value $f(x) = 1$ \textit{well-posed} or \textit{ill-posed}? Why?}
    \begin{solution}
      Let $f(x) = 1$. We will now check whether the solution $u(x, t) = f(te^{-x})$ satisfies the initial value
      \[ u(x, 0) = 1 \, \, \, \text{for } x \in \mathbb{R}.\]
  
      Plugging in $t=0$ to our solution, we get that \[ u(x, 0) = f(0e^{-x}) = f(0) \text{ for } x \in \mathbb{R} \]
  
      However, notice that to satisfy the initial condition, we must have that $u(x, 0) = f(x) = x$ for $x\in \mathbb{R}$. \\
  
      Clearly, \[ u(x, 0) = f(0) \neq f(x) \text{ for } x \in \mathbb{R}, \] and so the solution we found in (a) does not satisfy the initial value.
      \end{solution}
\end{enumerate}

\end{document}