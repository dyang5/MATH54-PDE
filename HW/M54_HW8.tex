\documentclass[11pt]{article}
\usepackage{graphics}
\usepackage{amsthm}
\usepackage{amsmath}
\usepackage{amssymb}
\usepackage{listings}
\usepackage{mathtools}
\usepackage[margin=1in]{geometry}
\usepackage[shortlabels]{enumitem}


\newenvironment{solution}
  {\renewcommand\qedsymbol{$\blacksquare$}\begin{proof}[Solution]}
  {\end{proof}}
  
\newcommand{\N}{\mathbb{N}}
\setlength\parindent{0pt}
\begin{document}

	\hrule
	\begin{center}
		{\Large Homework 8} \\ % Replace with the homework number
		\vspace{0.2cm}
		Partial Differential Equations, Spring 2023 \hfill David Yang % Replace with your name(s)
	\end{center}

\hrule

\vspace{1em}


\underline{HW 8 Problem} \\

\textbf{Consider the 1st order linear initial value PDE problem:
\[tu_t + u_x = 0 \, \, \, \text{for } t > 0, x \in \mathbb{R}\]
\[ u(x, 0) = f(x) \, \, \, \text{for } x \in \mathbb{R}.\]}

\begin{enumerate}[(a)]
    \item \textbf{Apply the Method of Characteristics. Your goal is to find a characteristic value $\xi(x, t)$ so that any function of the form $u(x, t) = f(\xi)$ satisfies the PDE.}
    \item \textbf{Set $f(x) = x$ as the initial value for the PDE given above. Show that the form of the solution you found in (a) does not satisfy this initial value. \\}
    
    \textbf{\textit{Remark: In fact, no solutions exist for this PDE that solve the initial value $u(x, 0) = x$. This PDE is ill-posed for $u(x, 0) = x$.}}
    \item \textbf{Set $f(x) = 1$ as the initial value for the PDE given above. Now let $\xi$ be the characteristic variable you found in (a). 
    For what values of the constants $a$ and $b$ does the function $u(x, t) = a + b\xi$ also solve the PDE and satisfy the initial value $u(x,0) = 1$?}\\

    \textbf{Is the PDE with the initial value $f(x) = 1$ \textit{well-posed} or \textit{ill-posed}? Why?}
\end{enumerate}

\end{document}