\documentclass[11pt]{article}
\usepackage{graphics}
\usepackage{amsthm}
\usepackage{amsmath}
\usepackage{amssymb}
\usepackage{listings}
\usepackage{mathtools}
\usepackage[margin=1in]{geometry}
\usepackage[shortlabels]{enumitem}


\newenvironment{solution}
  {\renewcommand\qedsymbol{$\blacksquare$}\begin{proof}[Solution]}
  {\end{proof}}
  
\newcommand{\N}{\mathbb{N}}
\setlength\parindent{0pt}
\begin{document}

	\hrule
	\begin{center}
		{\Large Homework 8} \\ % Replace with the homework number
		\vspace{0.2cm}
		Partial Differential Equations, Spring 2023 \hfill David Yang % Replace with your name(s)
	\end{center}

\hrule

\vspace{1em}


\underline{HW 9 Problems} \\

\underline{Chapter 4.1 Problem 5}

Solve by adapting the method of 3.1 (also reviewed in 4.1) to find a Fourier series solution for the PDE. \\

Fourier coefficients for functions defined on the interval $[0, l]$ are given on page 148. \\

\underline{Chapter 3.2 Problem 3(a)} \\


\underline{Chapter 3.2 Problem 6} \\

Also answer for this problem: Why does this formula makes sense?
In your (very brief) answer, you can relate this formula to a result you may have learned
in linear algebra if you studied orthogonal projection and orthogonal decomposition
If you did not discuss orthogonal projection in linear algebra, you can say "I did not
cover orthogonal projection in linear algebra."
\end{document}