\documentclass[11pt]{article}
\usepackage{graphics}
\usepackage{amsthm}
\usepackage{amsmath}
\usepackage{amssymb}
\usepackage{listings}
\usepackage{mathtools}
\usepackage[margin=1in]{geometry}
\usepackage[shortlabels]{enumitem}


\newenvironment{solution}
  {\renewcommand\qedsymbol{$\blacksquare$}\begin{proof}[Solution]}
  {\end{proof}}
  
\newcommand{\N}{\mathbb{N}}
\setlength\parindent{0pt}
\begin{document}

	\hrule
	\begin{center}
		{\Large Homework 8} \\ % Replace with the homework number
		\vspace{0.2cm}
		Partial Differential Equations, Spring 2023 \hfill David Yang % Replace with your name(s)
	\end{center}

\hrule

\vspace{1em}


\underline{HW 9 Problems} \\

\underline{Chapter 4.1 Problem 5} \\

\textbf{Consider heat flow in a rod of length $l$ where the heat is lost across the lateral boundary is given by Newton's law of cooling. The model is} 

\[
\begin{array}{c}
    u_t = ku_{xx} - hu, \,\,\, 0 < x < l, \\
    u = 0 \text{ at } x =0, \, x=l, \, \text{ for all } t > 0,\\
    u = f(x) \text { at } t=0, \, 0 \leq x \leq l,
\end{array}
\]

where $h > 0$ is the heat loss coefficient. \\

\begin{solution}
We will use the separation of variables method. Consider a separated solution of the form \[u(x, t) = y(x) g(t).\]

For this solution, we have that \[ u_t = y(x)g'(t) \text{ and } u_{xx} = y''(x) g(t).\]

Substituting these partials into our original PDE $u_t = ku_{xx} - hu$, we get that
\[ y(x)g'(t) = ky''(x)g(t) - hy(x)g(t).\]

Since neither $y(x)$ nor $g(t)$ are $0$, we can separate this equation further by dividing both sides by $y(x)g(t)$. This gives us
\[ \frac{g'(t)}{g(t)} = \frac{ky''(x) -hy(x)}{y(x)}.\]

For this equation to hold true for all values of $x \in (0, l)$ and $t>0$ is for them to evaluate to the same constant, and so we have
\[ \frac{g'(t)}{g(t)} = \frac{ky''(x) -hy(x)}{y(x)} = C\]

for some constant $C$. \\

Rewriting the above equation as two separate ODEs, we have that
\[
\begin{cases}
  g'(t) = Cg(t) \\
  y''(x) = \frac{C+h}{k}y(x)
\end{cases}
\]

We will begin by solving for $y(x).$ Note that, just like in our in-class example, nontrivial solutions $y(x)$ that satisfy the boundary conditions $y(0)=y(l) = 0$ will only occur when $C+h < 0$. Consequently, we can solve the ODE 
\[ y''(x) - \frac{(C+h)}{k}y(x) = 0\] when $\frac{C+h}{k} < 0$ to get the general form for $y(x)$: 
\[ y(x) = a\sin\left(\sqrt{\frac{-(C+h)}{k}}x\right) + b\cos\left(\sqrt{\frac{-(C+h)}{k}}x\right).\]

By the boundary conditions, we know that $y(0) = 0$ and $y(l) = 0.$ To satisfy $y(0) = 0$, we must have that
\[ y(0) = a\sin(0) + b\cos(0) = 0, \]
so $b = 0$. On the other hand, for $y(l) = 0$, we must have that
\begin{align*}y(l)&= a\sin\left(\sqrt{\frac{-(C+h)}{k}}l\right) + 0\cos\left(\sqrt{\frac{-(C+h)}{k}}l\right) \\
&= a\sin\left(\sqrt{\frac{-(C+h)}{k}}l\right) = 0.
\end{align*}

Since the sine function is $0$ at integer multiples of $\pi$, we know that \[\sqrt{\frac{-(C+h)}{k}}l = \pi n \] for some integer $n$. Equivalently, we have that 
\begin{align*}\sqrt{\frac{-(C+h)}{k}} &= \frac{\pi n}{l} \\
\frac{-(C+h)}{k} &= \left( \frac{\pi n}{l} \right)^2 \end{align*}

and so solving for $C$ gives us \[C = -k\left( \frac{\pi n}{l} \right)^2 - h.\]

Recall that our solution for $y(x)$ is \[y(x) =  a\sin\left(\sqrt{\frac{-(C+h)}{k}}x\right)\]
with $C = -k\left( \frac{\pi n}{l} \right)^2 - h$ for positive integers $n$. \\

Substituting our expression for $C$, we find that our general solution for $y$ is 
\[ y_n(x) = a_n \sin\left(\frac{\pi n}{l}x\right) \text { for positive integers } n.\]

We can now solve our other ODE for $g(t)$. We had that $g'(t) = Cg(t)$. Solving and using the general value for $C$, we have that $g(t) = e^{Ct}$ so 
\[ g_n(t) = e^{\left(-k\left( \frac{\pi n}{l} \right)^2 - h\right)t}\]

Thus, our product solutions that satisfy the original PDE and its boundary conditions are 
\[ u_n(x, t) = y_n(t) g_n(t) = a_ne^{\left(-k\left( \frac{\pi n}{l} \right)^2 - h\right)t} \sin\left(\frac{\pi n}{l} x\right). \]

Using superposition, our solution $u(x, t)$ is 

\[ \boxed{u(x, t) = \sum\limits_{n=1}^{\infty} a_ne^{\left(-k\left( \frac{\pi n}{l} \right)^2 - h\right)t} \sin\left(\frac{\pi n}{l} x\right)}\]

where $a_n$ are the Fourier coefficients defined on page 148: $a_n = \frac{2}{l}\int_0^l f(x) \sin\left(\frac{\pi n}{l} x \right).$

\end{solution}
\newpage

\underline{Chapter 3.2 Problem 3(a)} \\

\textbf{Let $f(x)=0$ for $0<x<1$ and $f(x)=1$ for $1<x<3$.}
\begin{enumerate}[a)] 
    \item \textbf{Find the first $4$ nonzero terms of the Fourier cosine series of $f$.}
  
\end{enumerate}

\begin{solution}
By definition, the Fourier cosine series of $f$ is \[ \frac{b_0}{2} + \sum\limits_{n=1}^{\infty} b_n \cos\left(\frac{n\pi x}{3}\right),\]

where \[b_0 = \frac{2}{3}\int_{0}^{3} f(x) \, dx = \frac{2}{3}\int_{1}^3 1 \, dx = \frac{4}{3}\] and 
\begin{align*} b_n &= \frac{2}{3} \int_0^3 f(x) \cos\left(\frac{n\pi x}{3}\right) \, dx \\
&= \frac{2}{3} \int_1^3 \cos\left(\frac{n\pi x}{3}\right) \, dx\end{align*}

for positive integers $n$. Simplifying further, we get that \begin{align*}b_n &= \frac{2}{3} \left[ \frac{3}{n\pi} \sin\left( \frac{n\pi x}{3}\right)\right]_1^3 \\ 
&=  \frac{2}{n\pi} \left( \sin(n\pi) - \sin\left( \frac{n\pi}{3}\right)\right)\end{align*}

Note that $\sin(n\pi) = 0$ for all integer $n$. Using this fact, we can simplify our general term to
\[ b_n = \frac{2}{n\pi} \left( -\sin\left( \frac{n\pi}{3}\right)\right)\]and plugging in a few values of $n$ to determine the first nonzero coefficients, we find that
\[b_0 = \frac{4}{3}\]
\[b_1 = \frac{2}{1\pi}\left( -\sin\left(\frac{\pi}{3}\right) \right) = -\frac{\sqrt{3}}{\pi}\]
\[b_2 = \frac{2}{2\pi}\left( -\sin\left(\frac{2\pi}{3}\right) \right) = -\frac{\sqrt{3}}{2\pi}\]
\[b_3 = \frac{2}{3\pi}\left( -\sin\left(\frac{3\pi}{3}\right) \right) = 0\]
\[b_4 = \frac{2}{4\pi}\left( -\sin\left(\frac{4\pi}{3}\right) \right) = \frac{\sqrt{3}}{4\pi}.\]

Thus, since the Fourier cosine series of $f$ is \[ \frac{b_0}{2} + \sum\limits_{n=1}^{\infty} b_n \cos\left(\frac{n\pi x}{3}\right),\] we find that the first $4$ nonzero terms of the Fourier cosine series of $f$ are
\[ \frac{b_0}{2} = \boxed{\frac{2}{3}} \]
\[ b_1\cos\left( \frac{1\pi x}{3}\right) = \boxed{-\frac{\sqrt{3}}{\pi}\cos\left(\frac{\pi x}{3}\right)}\]
\[ b_2\cos\left( \frac{2\pi x}{3}\right) = \boxed{-\frac{\sqrt{3}}{2\pi}\cos\left(\frac{2\pi x}{3}\right)}\]
\[ b_4\cos\left( \frac{4\pi x}{3}\right) = \boxed{\frac{\sqrt{3}}{4\pi}\cos\left(\frac{4\pi x}{3}\right)}\]

\end{solution}

\newpage

\underline{Chapter 3.2 Problem 6} \\

\textbf{If $c_n$ are Fourier coefficients of $f$ and $f_n$ is an orthonormal set, show that}

\[ \left(\sum\limits_{n=1}^N c_n f_n, \, f  - \sum\limits_{n=1}^N c_nf_n \right) = 0.\]

\textbf{Also answer for this problem: Why does this formula makes sense?
In your (very brief) answer, you can relate this formula to a result you may have learned
in linear algebra if you studied orthogonal projection and orthogonal decomposition.}

\begin{solution}
By the linearity property of inner products, we know that 
\[ \left(\sum\limits_{n=1}^N c_n f_n, \, f  - \sum\limits_{n=1}^N c_nf_n \right)\]
\[ =\left(\sum\limits_{n=1}^N c_n f_n, \, f\right) - \left( \sum\limits_{n=1}^N c_n f_n,\sum\limits_{n=1}^N c_nf_n \right).\]

Notice that by definition, $f = \sum\limits_{n=1}^{\infty} c_nf_n$, so \begin{align*} f &= \sum\limits_{n=1}^{\infty} c_nf_n \\
&= \sum\limits_{n=1}^{N} c_nf_n + \sum\limits_{n=N+1}^{\infty} c_nf_n. \end{align*}

Substituting this back into our inner product expression above, we get

\[ \left(\sum\limits_{n=1}^N c_n f_n, \, f\right) - \left( \sum\limits_{n=1}^N c_n f_n,\sum\limits_{n=1}^N c_nf_n \right) \]
\[ = \left(\sum\limits_{n=1}^N c_n f_n, \, \sum\limits_{n=1}^{N} c_nf_n + \sum\limits_{n=N+1}^{\infty} c_nf_n\right) - \left( \sum\limits_{n=1}^N c_n f_n,\sum\limits_{n=1}^N c_nf_n \right). \]

Using the linearity property once again on the first term, we find that this expression is simply

\[ = \left(\sum\limits_{n=1}^N c_n f_n, \, \sum\limits_{n=1}^{N} c_nf_n\right)  +\left( \sum\limits_{n=1}^N c_nf_n , \sum\limits_{n=N+1}^{\infty} c_nf_n\right) - \left( \sum\limits_{n=1}^N c_n f_n,\sum\limits_{n=1}^N c_nf_n \right). \]

Grouping our terms and simplifying, we get that
\[ \left(\sum\limits_{n=1}^N c_n f_n, \, \sum\limits_{n=1}^{N} c_nf_n\right)  +\left( \sum\limits_{n=1}^N c_nf_n , \sum\limits_{n=N+1}^{\infty} c_nf_n\right) - \left( \sum\limits_{n=1}^N c_n f_n,\sum\limits_{n=1}^N c_nf_n \right)\]
\[ = \left(\left(\sum\limits_{n=1}^N c_n f_n, \, \sum\limits_{n=1}^{N} c_nf_n\right) - \left(\sum\limits_{n=1}^N c_n f_n, \, \sum\limits_{n=1}^{N} c_nf_n\right)\right) + \left( \sum\limits_{n=1}^N c_nf_n , \sum\limits_{n=N+1}^{\infty} c_nf_n\right).\]
\[ = \left( \sum\limits_{n=1}^N c_nf_n , \sum\limits_{n=N+1}^{\infty} c_nf_n\right).\]

However, since $\{f_n\}$ is an orthonormal set, $(f_i, f_j) = 0$ when $i \neq j$, and so this term simplifies to $0$.\\

Thus, we have that \[ \left(\sum\limits_{n=1}^N c_n f_n, \, f  - \sum\limits_{n=1}^N c_nf_n \right) = 0\]

as desired. \\

\textit{Note: This formula makes sense as we apply the fact that $f_{n}$ is an orthonormal set. If we subtract off the first $N$ terms in the Fourier series, we are left with a sum that is ``orthogonal" to our original sum, so we should find that the inner product is $0$.   
I think we may have briefly covered orthogonal projections in linear algebra; perhaps we could
imagine this as an orthogonal projection of an orthogonal subspace onto another one (which should be $0$). I am not too sure about this interpretation. }
\end{solution}
\end{document}