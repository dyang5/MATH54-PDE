\documentclass[11pt]{article}
\usepackage{graphics}
\usepackage{amsthm}
\usepackage{amsmath}
\usepackage{amssymb}
\usepackage{listings}
\usepackage{mathtools}
\usepackage[margin=1in]{geometry}
\usepackage{hyperref}
\usepackage{xcolor}


\hypersetup{
    colorlinks,
    linkcolor={blue!50!black},
    citecolor={blue!50!black},
    urlcolor={blue!80!black}
}
\newenvironment{solution}
  {\renewcommand\qedsymbol{$\blacksquare$}\begin{proof}[Solution]}
  {\end{proof}}

  \newtheorem{theorem}{Theorem}[section]
  % \newtheorem{exercise}[theorem]{Exercise}
  \newtheorem{probleminternal}{Problem}

  \NewDocumentEnvironment{prob}{o}{%
    \IfValueT{#1}{%
      \renewcommand{\theprobleminternal}{#1}%
      \addtocounter{probleminternal}{-1}%
    }%
    \probleminternal%
  }
  {\endprobleminternal}

  \NewDocumentEnvironment{exercise}{m}{
    \newtheorem*{ex#1}{Exercise #1}
    \begin{ex#1}
}{
    \end{ex#1}
}

\newcommand{\N}{\mathbb{N}}
\setlength\parindent{0pt}

\begin{document}


\title{\large	{\textsc{Typos in Logan's \textit{Applied Partial Differential Equations, 3rd ed.} Solutions}}}
\author{\textsc{David Yang}}
\date{}
\maketitle

\setcounter{tocdepth}{2}

\tableofcontents
\newpage
\vspace{1em}

\section{Chapter 1}
\setcounter{subsection}{1}
\subsection{Chapter 1, Section 2}
\begin{exercise}{7} Solve the initial boundary value problem \[u_t+cu_x = \lambda u, \, \, x, t > 0 \]
\[u(x,0)=0, \, x>0, \, u(0,t)=g(t), \, t > 0.\]
\end{exercise}

\begin{solution}
We should find that $\phi(t) = e^{-\lambda t/c} g(\textcolor{red}{-}t/c)$ (notice the negative in the exponent). This follows from the fact that if $\lambda(-ct)e^{\lambda t} = g(t)$, then substituting $t = -t/c$ gives us the negative in the exponent.
This gives us $u(x, t) = g(t-x/c)e^{-\lambda x/c},$ in $0 \leq x < ct$, which matches the solutions.
\end{solution}

\begin{exercise}{12} Find a formula that implicitly defines the solution$ u = u(x, t)$ of the initial value problem for the reaction-advection equation
\[ u_t + cu_x = -\frac{\alpha u}{\beta + u}, \, \, x \in \mathbb{R}, \, t > 0, \] 
\[ u(x,0) = f(x), \, \, x \in \mathbb{R} \]
  Here, $v, \alpha, \beta$ are positive constants. Show from the implicit formula that you can always solve for $u$ in terms of $x$ and $t$.
  \end{exercise}
  
  \begin{solution}
  \textcolor{red}{$f(x)$ should be $f(x-ct)$, as we need to change back to $x-t$ coordinates.}
  \end{solution}

\subsection{Chapter 1, Section 3}
\begin{exercise}{2}
  Let $u = u(x, t)$ satisfy the heat flow model 
  \[ u_t =ku_{xx}, \, \, 0<x<l, \, t>0 \]
  \[ u(0,t) = u(l,t) = 0, \, \, t > 0,\]
  \[ u(x,0)=u_0(x), 0 \leq x \leq l.\] 
  
  Show that
  \[ \int_0^l u(x, t)^2 \, dx \leq \int_0 u_0(x)^2 \, dx, \, t \geq 0. \]
  
  Hint: Let $E(t) = \int_0^l u(x, t)^2 \, dx$ and show that $E'(t) \leq 0$. What can be said about $u(x, t)$ if $u_0(x) = 0$?
\end{exercise}

\begin{solution}
We have that $E(t) \leq E(0) = \int_0^l u_0(x)^{\textcolor{red}{2}} \, dx.$ This does not affect the solution, though, as if $u_0 \equiv 0$, then $E(0) = 0$ still.

\end{solution}

\begin{exercise}{6}
Heat flow in a metal rod with a unit internal heat source is governed by the problem
\[u_t=ku_{xx}+1, \, 0<x<1, \, t>0, \]
\[u(0,t)=0, \, u(1,t)=1, \, t>0. \]
What will be the steady-state temperature in the bar after a long time? Does it matter that no initial condition is given?
\end{exercise}

\begin{solution}
  The answer should be \[u(x) = -\frac{1}{2k}x^2 \textcolor{red}{+} \left(1+\frac{1}{2k}\right) x.\]

  Originally, there is a $=$ sign rather than a $+$.
\end{solution}

\setcounter{subsection}{4}
\subsection{Chapter 1, Section 5}

\begin{exercise}{5}The total energy of the string governed by equation (1.37) with boundary conditions (1.40) is defined by \[E(t) = \int_{0}^{\ell} \left( \frac{1}{2} \rho_0 u_t^2 + \frac{1}{2} \tau_0 u_x^2 \right) \, dx.\]
Show that the total energy is constant for all $ t \geq 0$. Hint: Multiply (1.37) by $u_t$ and note that $(u_t^2)_t = 2u_tu_{tt}$ and $(u_tu_x)_x = u_tu_{xx} + u_{tx}u_x.$ Then
show that \[\frac{d}{dt} \int_{0}^{\ell} \rho_0 u_t^2 \, dx= [\textcolor{red}{2}\tau_0u_tu_x]_0^\ell - \frac{d}{dt}  \int_{0}^{\ell} \tau_0 u_x^2 \, dx.\]
\end{exercise}

\begin{solution}
The hint should include an extra factor of $2$ (colored in \textcolor{red}{red}).
\end{solution}

\begin{exercise}{8}
  At the end $(x = 0)$ of a long tube $(x \geq 0)$ the density of air changes according to the formula $\tilde{\rho}(0,t) = 1 - \cos 2t$ for $t \geq 0,$ 
  and $\tilde{\rho}(0,t) = 0$ for $t < 0$. Find a solution to the wave equation in the domain $x>0, -\infty<t< \infty$, in the form of a 
  right-traveling wave that satisfies the given boundary condition. Take $c = 1$ and plot the solution surface.
\end{exercise}

\begin{solution}
We should have that \[\tilde{\rho}(0,t) = F(-ct) = 1- \cos(\textcolor{red}{2}t).\] This tells us that \[F(t) = 1-\cos(\textcolor{red}{2}(t - x/c)). \]

Notice that the factors of $2$ are inside the $\cos$ term rather than outside (as a coefficient).
\end{solution}

\section{Chapter 2}

\setcounter{subsection}{4}
\subsection{Chapter 2, Section 5}

\begin{exercise}{3}
Using Duhamel's principle, find a formula for the solution to the initial value problem for the convection equation
\[u_t +cu_x =f(x,t), \, x \in \mathbb{R}, \, t>0; \, \, \, u(x,0)=0, \, x \in \mathbb{R}.\]
\end{exercise}


\begin{solution}
The solution contains no typos, but should be split into two solutions (pertaining to Exercises 3 and 4.)
\end{solution}

\begin{exercise}{4}
Solve the problem 
\[u_t +2u_x = xe^{-t}, \, x \in \mathbb{R}, \, t>0; \, \, \, u(x,0)=0, \, x \in \mathbb{R}.\]
\end{exercise}

\begin{solution}
The given answer of \[u(x, t) = -(x - 2t)(e^{-t} - 1) - 2te^{-t} + 2(1 - e
^{-t})\] is correct, but may be more clear in its simplest form, which is the result one should get after integration by parts:
\[ u(x, t) = (-x-2)e^{-t} + x - 2t + 2.\]
\end{solution}

\appendix

\newpage

\section{Completed Problems}

The following list consists of problems I have done. The list is non-exhaustive, but contains problems that I did not find typos in.
\subsection{Chapter 1}
\begin{itemize}
  \item Section 1.1: 1-11
  \item Section 1.2: 1-11
  \item Section 1.3: 2, 3, 4, 6, 9
  \item Section 1.5: 3, 4, 5, 8, 9
  \item Section 1.7: 6
\end{itemize}

\subsection{Chapter 2}
\begin{itemize}
  \item Section 2.5: 1-5
\end{itemize}

\end{document}
